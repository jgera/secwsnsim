\begin{table}[H]
 \centering
\begin{footnotesize}
\begin{tabular}[t]{|l|p{11cm}|}\hline
\textbf{Opera��o} & \textbf{Descri��o} \\\hline
\textbf{newRouteDiscovery()} & envia a primeira mensagem de Route Request para
os vizinhos com a
informa��o do seu identificador e assinada com a chave privada, inicia o
temporizador
para recolha das mensagens de feedback;\\\hline
\textbf{processFDBKMessage()} & verifica a validade da mensagem de feedback
quanto � origem e
quanto � integridade, com base na cadeia de MAC?s e na informa��o do caminho
per-
corrido guardando-a para posterior processsamento;\\\hline
\textbf{startComputeRoutingInfo()} & recolhida a informa��o de feedback limitada
pelo tempo definido
no temporizador, procede-se ao c�lculo das tabelas de encaminhamento
correlacionando
a informa��o enviada pelos diversos n�s, garantindo o mais cedo poss�vel a
defesa contra
diversos ataques, como por exemplo, o Wormhole. O c�lculo das tabelas �
efectuado com
recurso ao algoritmo de Dijkstra;\\\hline
\textbf{sendRouteUpdateMessages()} & uma vez obtidas as tabelas de
encaminhamento, inicia-se o pro-
cesso de envio das tabelas para cada um dos n�s seguindo uma ordem dos n�s mais
pr�ximos para os mais afastados aproveitando os caminhos j� definidos para os
primeiros
para chegar aos sensores mais afastados da rede.\\\hline
\end{tabular}
\caption{Tabela de opera��es princ�pais da esta��o
base}\label{tab:operacoes_estacao_base}
% \end{tiny}
\end{footnotesize}
\end{table}
