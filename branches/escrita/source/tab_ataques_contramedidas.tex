%tabela de ataques e contra-medidas
\begin{table}[H]
 \centering
\begin{tiny}
\begin{tabular}[t]{l|l|p{4cm}}
\hline
  Modelos & Ataque & Contramedidas\\ \hline\hline
\multirow{1}{*}{Dolev-Yao}
& Ataque ao meio de comunica��o & Criptografia sim�trica, \textit{One Way Hashing} \\\cline{2-3}
\hline 
%
\multirow{4}{*}{Organiza��o e Descoberta da Rede}
 & Falsifica��o de informa��o de Routing & Autentica��o, \textit{One Way Hashing}\\\cline{2-3}
%%
 & Ataques de \textit{Rushing} & Selec��o aleat�ria de RREQ, autentica��o, verifica��o bidirectional
\\ \cline{2-3}
\hline
%
\multirow{4}{*}{Estabelecimento de Rotas}
 & HELLO flooding & Autentica��o com verifica��o bidirectional(\textit{acknowledge})\\\cline{2-3}
%% 
& Ataques \textit{Sinkhole} & Autentica��o, Distribui��o de chaves \textit{pairwise} \\\cline{2-3}
%% 
& Ataques \textit{Wormhole} & \textit{Packet leaches}, MAC\\\cline{2-3}
%% 
& Ataques \textit{Sybil} & Distribui��o de chaves \textit{pairwise}, selec��o aleat�ria de canais
de r�dio \\
 \hline
%
\multirow{1}{*}{Manuten��o de Rotas} & Ataques de \textit{Backhole} & Defini��o de temporizadores e
mecanismos de confirma��o (ACK) autenticados\\
\hline
%
\multirow{2}{*}{Modelo de Intrus�o}
& Intrus�o& Encaminhamento multi-rota; \textit{One Way Hashing} \\\cline{2-3}
%%
& Replica��o& Certifica��o central; Autentica��o; N�s vizinhos como testemunhas\\\cline{2-3}
\hline

\end{tabular} 
\caption{Tabela de Ataques \textit{vs} Contramedidas}\label{tab:tabela_ataques_contramedidas}
\end{tiny}
\end{table}
