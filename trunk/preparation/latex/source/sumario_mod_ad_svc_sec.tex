\subparagraph{Sum�rio}
    Mediante as vulnerabilidades de uma RSSF, � necess�rio estabelecer um modelo
de advers�rio com vista a poder mapear as capacidades e tipologias de ataques
deste em mecanismos de seguran�a com o prop�sito de lhes poder resistir ou
mitiga-los. O modelo de Dolev-Yao � o modelo de facto quando se trata da an�lise
de amea�as a redes , em que o meio de comunica��o est� sobre controlo do
advers�rio. No entanto, tratando-se de RSSF, este modelo per si n�o se vislumbra
suficiente para abarcar todas as problem�ticas de seguran�a a que este tipo de
redes est� sujeita. Surge assim a necessidade de, face � inseguran�a que cada n�
da rede representa, estender este modelo acrescentando-lhe um modelo de
intrus�o.
    Perante a exposi��o das RSSF, os ataques que se podem desencadear podem ser
diferentes dos observados nas redes convencionais sendo assim necess�rio
considerar outras tipologias de ataques. Assim, podemos classificar os ataques
como activos e passivos \cite{Stallings2005} e os atacantes como internos e
externos\cite{Karlof2003}.
Nestes ultimos ainda se pode classificar quanto aos recursos usados como
\textit{sensor-class} ou \textit{laptop-class}\cite{Karlof2003}. Os ataques que
se consideram para o estudo e
relacionados com as RSSF s�o: falsa informa��o de encaminhamento,
\textit{blackhole},\textit{sinkhole}, \textit{wormhole} e  \textit{sybil
attack}\cite{Douceur2002}.