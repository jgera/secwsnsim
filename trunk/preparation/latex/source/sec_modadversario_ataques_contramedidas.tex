\section{Modelo de Advers�rio, Ataques ao Encaminhamento e
Contra-medidas} \label{sect:sec_mod_adversario_ataq_contramedidas}

S�o v�rios os ataques que se pode direccionar contra a pilha da arquitectura de
uma RSSF. Cada uma das camadas da pilha tem vulnerabilidades pr�prias das
fun��es que desempenha. No escopo deste trabalho o foco � a seguran�a ao n�vel
dos protocolos de encaminhamento, representada pela camada de transporte, da�
nesta sec��o se apresentar com algum detalhe as fases tradicionais de um
protocolo de encaminhamento em MANETs\cite{Corson1999} e em particular em
RSSF.\\

    Os protocolo de encaminhamento em redes de sensores, de uma forma geral
dividem-se em tr�s fases: descoberta dos caminhos, selec��o dos caminhos,
manuten��o da comunica��o pelos caminhos seleccionados. Importa neste momento
real�ar o facto de que os ataques a um algoritmo de encaminhamento normalmente
exploraram as vulnerabilidades de cada uma destas fases de forma especifica.
Da�, em seguida se proceder � associa��o dos ataques espec�ficos que se podem
desencadear em cada fase e como estes podem ser mitigados aplicando determinados
mecanismos de seguran�a como contra-medidas.

%\subsection{Ataques � organiza��o da rede e descoberta de n�s} \label{sect:subsec_ataq_org_rede}
Ap�s a descoberdos n�s vizinhos � necess�rio recolher informa��o para a constru��o das tabelas
de encaminhamento, isto nos protocolos do tipo \textit{table-driven}\cite{al-karaki_routing_2004}.
No entanto, em protocolos do tipo \textit{on-demand}\cite{al-karaki_routing_2004} esta fase �
desencadeada em cada in�cio de transmiss�o.
\paragraph*{\textbf{Falsifica��o de Informa��o de Encaminhamento}}
Este ataque tem impacto na forma��o da rede e na descoberta dos n�s. Induz a cria��o de entradas
incorrectas nas tabelas de encaminhamento, podendo tamb�m fazer com que estas fiquem lotadas e
inv�lidas. Nos protocolos \textit{on-demand}, o impacto pode ser menor, uma vez que obriga o 
atacante a injectar informa��o errada a cada ciclo de transmiss�o.
\paragraph*{\textbf{\textit{Rushing Attacks}}}
Outro ataque nesta fase � o \textit{rushing attack}\cite{Rushing_attacks_perrig} que � definido pela
explora��o, por parte do atacante, de uma janela de oportunidade para responder a um pedido de
caminho para um destino. Este ataque � efectivo quando o protocolo permite aceitar a primeira
resposta que recebe (ex: AODV\cite{Perkins1999}). Explorando isto, o atacante � sempre um
candidato a ser o pr�ximo encaminhador, uma vez que n�o respeita temporizadores nem condi��es de
resposta, podendo influenciar as rotas.
\subsubsection{Contra-medidas}
A aplica��o de mecanismos de autentica��o no protocolo de encaminhamento faz com que ataques de
falsifica��o de informa��o seja minimizados. Os n�s da rede podem partilhar chaves sim�tricas como
forma de autenticar as mensagens de dados e controlo do encaminhamento (RREQ e RREP). Desta forma, o
atacante como n�o possui a chaves necess�rias para a comunica��o, n�o poder� participar no
protocolo.

Para fazer face a ataques de \textit{rushing}, alguns autores \cite{Rushing_attacks_perrig}
apresentam dois mecanismos de defesa: reenvio aleat�rio de RREQ ()\textit{randomized RREQ
forwarding}) e detec��o segura (\textit{secure detection}). No primeiro mecanismo, cada n� guarda um
conjunto de mensagens RREQ escolhendo depois aleat�riamente um para reenviar. Ainda assim, pode ser
seleccionada uma mensagem RREQ maliciosa, da� a exist�ncia do segundo mecanismo, que proporciona a
troca de mensagens autenticadas entre dois n�s garantindo que as mensagens pertencem a n�s
leg�timos. 
 