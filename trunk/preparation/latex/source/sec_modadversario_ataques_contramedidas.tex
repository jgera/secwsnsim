\section{Modelo de Advers�rio, Ataques ao Encaminhamento e
Contra-medidas} \label{sect:sec_mod_adversario_ataq_contramedidas}
%REVISTO
\subsection{Arquitectura de Servi�os de Seguran�a em
RSSF} \label{sect:subsec_arq_security_wsn}
Num sistema seguro, � necess�rio que a seguran�a esteja integrada em cada um dos seus componentes, 
n�o se confinando a um componente isolado do sistema\cite{sec_in_wsn_perrig}.
Assim, nesta sec��o apresenta-se, introdutoriamente, alguns requisitos de seguran�a de uma RSSF e 
alguns servi�os de seguran�a, que foram implementados com o objectivo de
representarem um ponto de partida para a garantia de propriedades de seguran�a, a quando do desenho
de RSSF seguras.
\subsubsection{Requisitos de seguran�a de uma RSSF}
Os requisitos de seguran�a de uma RSSF podem variar consoante as especificidades da aplica��o que a
rede visa suportar. No entanto, apresentam-se, de forma gen�rica, os principais requisitos de
seguran�a de uma RSSF\cite{sec_in_wsn_perrig}:
\begin{description}\descvspace
 \item[Autentica��o]
Devido ao meio de comunica��o ser partilhado, � necess�rio recorrer � autentica��o para garantir a
detec��o de mensagens alteradas ou injectadas no sistema por participantes n�o
 autorizados\cite{sec_in_wsn_perrig}. Note-se que a implementa��o de criptografia assim�trica
contribui para a garantia desta propriedade, mas ainda existe muito esfor�o a desenvolver neste
campo dadas as limita��es das RSSF e as exig�ncias computacionais e energ�ticas destes
mecanismos.
 \item[Confidencialidade]
Sendo a RSSF uma infraestrutura baseada fundamentalmente na dissemina��o de dados recolhidos a
partir de sensores que se encontram distribuidos em ambiente n�o controlado e, normalmente, de
f�cil acesso,  � necess�rio garantir a confidencialidade dos dados que circulam na rede.
Assim, o uso de mecanismos de criptografia � o mais indicado para este tipo de protec��o.Desta
forma, � necess�ria a utiliza��o de algoritmos de encripta��o fi�veis (ex:
AES\footnote{\textit{Advenced Encryption System algorithm}}\cite{Stallings2005},
ECC\cite{Stallings2005}) para garantir
um determinado n�vel de seguran�a, para isso existe a necessidade de partilhar chaves de sess�o por
todos os \textit{end-points} e como tal deve-se recorrer a esquemas de distribui��o de
chaves\cite{eschenauer2002}.
 \item[Disponibilidade]
Entende-se por disponibilidade de uma rede, a garantia de que esta funciona efectivamente
durante o seu tempo de opera��o. Os ataques por nega��o de servi�o (Denial of Service -
DoS)\cite{Hu2005} s�o os mais frequentes para diminuir a disponibilidade de uma rede. Ent�o, para
al�m de mecanismos que evitem a nega��o de servi�o, � necess�rio garantir que a degrada��o da
rede (ex: na presen�a de um ataque ) seja controlada e que v� sendo proporcional ao n�mero de n�s
comprometidos.
 \item[Integridade]
A integridade garante que os dados recebidos por um n� n�o foram alterados, por um
advers�rio, durante a transmiss�o. Em alguns casos esta propriedade � garantida juntamente com a
autentica��o, usando mecanismos que permitam garantir ambos numa s� opera��o. Por exemplo, o uso de
CMAC's\cite{Stallings2005} � vulgar uma vez que permite autenticar (uso de critografia sim�trica)
a mensagem e para garantir a integridade da mensagem.\cite{SPINS}.
\item[Frescura]
A frescura de uma mensagem implica que estes sejam recentes, garantindo que esta n�o � antiga
e n�o foi reenviada por um qualquer advers�rio. \cite{SPINS,Luk2007d} Podem-se considerar dois tipos
de frescura: frescura fraca (garantindo ordem parcial e sem informa��o do desvio de tempo, usada
para as medi��es dos sensores) e frescura forte (garante ordem total em cada
comunica��o permitindo estimar o atraso, usada para a sincroniza��o de tempo).
 \end{description}
\subsubsection{Servi�os de Seguran�a}
Alguns servi�os de seguran�a t�m vindo a ser desenvolvidos para as RSSF com vista a garantir a
seguran�a ao n�vel da comunica��o (ex: criptografia, assinaturas, \textit{digests}). Estes servi�os
permitem que o arquitecto de sistemas se centre em outras problem�ticas relacionadas com o
comportamento dos protocolos face outros ataques. Apresentam-se de seguida alguns servi�os mais
comuns:
\begin{description}\addtolength{\itemsep}{-0.5\baselineskip}
\item[TinySec\cite{Karlof2004}]
TinySec � uma arquitectura de seguran�a para protec��o ao n�vel de liga��o de
dados em RSSF. O objectivo principal, para o qual foi desenhado, � providenciar
um n�vel adequado de seguran�a com o m�nimo consumo de recursos. Os servi�os de
seguran�a  disponibilizados s�o: autentica��o de dados (com a utiliza��o de
\textit{Message Authentication Codes}(MAC)\cite{Stallings2005}, no caso CBC-MAC\footnote{Cipher
Block Chaining - Message Authentication Code (CBC-MAC))}) e confidencialidade (CBC-MAC).
N�o implementa nenhum mecanismo que garanta a frescura das mensagens, tornando-o vulner�vel a
ataques de \textit{replaying}).
\item[MiniSec\cite{Luk2007d}]
Minisec � uma camada de rede concebida para possuir baixo consumo de energia, melhor que o TinySec,
e alta seguran�a. Uma das caracteristicas principais, que a tornam mais eficiente, � o uso do modo
\textit{offset codebook} (OCB)\cite{Stallings2005}  para encripta��o de blocos. O que permite numa
�nica passagem autenticar e encriptar os dados, sem com isso aumentar o tamanho da mensagem em
claro, contribuindo para um menor consumo de energia. Esta arquitectura tem dois modos de opera��o:
uma baseado paracomunica��o \textit{unicast} (MINISEC-U) e outro para \textit{broadcast}
(MINISEC-B). Sendo que a segunda n�o necessita de manter o estado (sincroniza��o de contadores) por
cada emissor por forma a proteger o reenvio escalando para grandes redes.
\item[SPINS\cite{SPINS}]
Conjunto de protocolos de seguran�a, constitu�do por dois componentes
principais SNEP\footnote{Secure Network Encryption Protocol} \cite{SPINS} e
${\mu}$TESLA \cite{SPINS,Luk2006}. O primeiro, fornece servi�os de autentica��o e
confidencialidade entre dois pontos de comunica��o, encriptando as mensagens (com
o modo CTR\footnote{\textit{Counter Mode}}) e protegendo-as com um MAC (autentica��o com CBC-MAC). O
SNEP gera diferentes chaves, de encripta��o, que derivam de uma chave mestra partilhada entre os
dois n�s, com umcontador de mensagens para garantir a frescura. O segundo componente,o
${\mu}$TESLA\cite{SPINS,Luk2006}, � um servi�o de autentica��o de \textit{broadcast}, que evita a
utiliza��o de mecanismos, mais exigentes, de criptografia assim�trica, recorrendo a critografia
sim�trica, autenticando as mensagens com um MAC, 
\item[Norma IEEE802.15.4\cite{ietf_802154}]
Esta norma define a especifica��o da camada f�sica e de controlo de acesso ao meio das redes
pessoais de baixa pot�ncia (\textit{LRPAN}\footnote{Low Rate Personal Area Networks}). Foca-se
essencialmente na comunica��o entre dispositivos relativamente pr�ximos, sem a
necessidade de uma infrastrutura de suporte, explorando o m�nimo de consumo de energia. � a norma
que j� se encontra implementada em algumas plataformas das RSSF. (ex: Micaz\cite{micaz}).
Esta norma, especifica alguns servi�os de seguran�a\cite{zigbee_802154}, representam uma primeira
linha de protec��o contra ataques � infraestrutura. Estes mecanimos s�o os seguintes: i) Cada
dispositivo mant�m uma lista de controlo de acessos (ACL) dos dispositivos confi�veis
filtrando comunica��es n�o autorizadas; ii) Encripta��o de dados, partilha de uma chave
criptogr�fica entre os intervenientes na comunica��o; iii) Servi�o de integridade de cada
\textit{frame}, adicionando a cada frame uma \textit{Message Integrity Code}
(MIC)\cite{Stallings2005}; iv) Garantia de frescura de mensagens (\textit{Sequential Freshness}),
utilizando contadores de frames e de chaves.
\item[ZigBee\cite{zigbee_802154,zigbee}]
Com a norma 802.15.4 orientada para as duas camadas mais baixas da pilha de
protocolos das RSSF (f�sica e MAC), a norma ZigBee define as especifica��es para
a camada de rede e aplica��o. J� incorpora alguns servi�os de seguran�a, nomeadamente: i) Frescura,
mantendo contadores associados a cada chave de sess�o, que s�o reiniciados em cada mudan�a de chave;
ii) Integridade, com op��es de integridade de mensagens que v�o desde os 0 aos 128 bits de
verifica��o; ii) Autentica��o, ao n�vel de rede e ao n�vel de liga��o de dados;
iv) Confidencialidade, com o algoritmo AES\cite{Stallings2005} com 128 bits.
Esta arquitectura utiliza um \textit{trusted center} para gest�o da seguran�a na rede,
implementando um coordenador de rede ZigBee. Este, acreditado por todos os n�s da rede, pode
desempenhar tr�s fun��es: i) Autentica��o de n�s que pretendem participar na rede; ii) Manuten��o e
distribui��o de chaves; iii) Providenciar seguran�a ponto-a-ponto entre n�s da rede.
\end{description}
\subsection{Modelo de Advers�rio} \label{sect:sec_mod_adversario_serv_seg}
Quando se tratam de quest�es de seguran�a, qualquer se seja o seu dom�nio,
existe uma primeira pergunta que cumpre fazer: ``quais s�o as amea�as/ataques a
que est� sujeito o objecto que se pretende manter seguro?''. Esta pergunta
possibilita, desde logo, encetar uma caminhada que visa a identifica��o de quais
os poss�veis atacantes, que capacidades estes possuem, quais os meios e modos
que estes podem utilizar e em que momento o ataque se pode desencadear. Esta
abordagem, um tanto ou quanto generalista, � suficiente para ilustrar a forma
como se pretende orientar o estudo e com isto apresentar, nas mesmas vertentes,
o modelo de advers�rio que enforma este trabalho.
 \subsubsection{Modelo de Dolev-Yao}\label{sect:subsec_dolev_yao}
Um dos modelos de advers�rio mais conhecidos, quando se trata de
an�lise formal de protocolos seguros, � o modelo de Dolev-Yao \cite{Dolev1983}. 
Assim, neste modelo, � considerado que a rede est� sobre o dom�nio do
advers�rio, que perante este facto pode extrair, reordenar, reenviar, alterar e
apagar as mensagens que circulam entre quaisquer dois principais legitimos. Com
esta assump��o, entende-se portanto, que o advers�rio transporta a mensagem e
com isso adopta um ataque do tipo \textit{man-in-the-middle}\cite{Stall2005},
com
comportamento incorrecto, que o leva a poder alterar o destinat�rio, atribuir
uma falsa origem, analisar o tr�fego ou alterar as mensagens. Este
funcionamento, entenda-se, n�o � comparado � intrus�o mas sim � intercep��o de
mensagens que pode ser mitigado usando mecanismos de criptografia.

As tipologias de ataque, consideradas pelo o modelo de advers�rio de Dolev-Yao
s�o  instanciadas pela norma X800 \cite{ITU-T1991} que pretende
normalizar
uma arquitectura de seguran�a para o modelo OSI, oferecendo uma abordagem
sistem�tica para o desenho de sistemas seguros. Esta norma considera a seguran�a
sobre tr�s aspectos: ataque, mecanismo e servi�o de
seguran�a\cite{Stall2005}. O
primeiro refere-se � forma usada para comprometer um sistema, por exemplo,
alterando ou tendo acesso n�o autorizado autorizado a dados desse sistema. Na
literatura, algumas vezes usam-se os temos ataque e amea�a para denominarem o
mesmo efeito, no entanto recorrendo ao RFC 2828 \cite{RFC2828} podemos
definir amea�a como uma potencial viola��o de seguran�a, ou seja � apenas uma
possibilidade que pode ser usada para desencadear um ataque explorando uma
vulnerabilidade; no caso do ataque, trata-se da explora��o inteligente de uma ou
mais amea�as que resultam na viola��o com sucesso de um sistema que se pretendia
seguro. O segundo aspecto considerado, na norma X.800, s�o os mecanismos de
seguran�a, que se entende como o processo que permite detectar, prevenir ou
recuperar de uma ataque � seguran�a (ex: encripta��o, controlo de acesso,
assinatura digital)\cite{Stall2005}. Por fim, o terceiro aspecto define os
servi�os
que, fazendo uso de um ou mais mecanismos de seguran�a, permitem resistir a
ataques dirigidos a determinada fonte de informa��o, quer seja durante o
processamento ou durante a comunica��o.
 \subsubsection{Modelo de Intrus�o em RSSF}\label{sect:subsec_intrusao}
Considerando o estudo de seguran�a numa RSSF, e dada a sua exposi��o
natural, nomeadamente a f�sica, colocando cada sensor ao alcance de um qualquer
advers�rio, torna relevante a considera��o de novos modelos de ataque.
Considerando que cada rede pode ser constitu�da por milhares de sensores, cada
um deles � um ponto de ataque, na impossibilidade de se proteger ou monitorizar
todos os sensores instalados\cite{Perriga}. Assim
as RSSF v�m-se sujeitas a um modelo de advers�rio que difere das redes com/sem
fios convencionais. Um advers�rio pode estar perto da rede e ter acesso aos
sensores e com isto ``roubar'' um ou parte dos sensores da rede com vista a
explorar os segredos ou material criptogr�fico usados para a comunica��o.
Podemos ent�o tipificar estes ataques como sendo por intrus�o. 
Este tipo de ataques podem ser definidos por ataques desde o n�vel
MAC\cite{Xiao2006} at� ao n�vel de intrus�o f�sica  em que um actor externo,
tendo acesso a um ou m�s sensores leg�timos, descobre os segredos criptogr�ficos
permitindo-lhe replicar\cite{Parno2005} os segredos para sensores maliciosos,
que depois de introduzidos podem agir de forma coordenada comprometendo a rede.
Conseguida a intrus�o, o atacante pode induzir nos sensores leg�timos
comportamentos incorrectos baseados na informa��o falsa introduzida pelos
sensores maliciosos, influenciando o processo de encaminhamento (denominados de
ataques ao encaminhamento).  Note-se, por exemplo, que estes ataques t�m
caracter�sticas que os tornam dif�ceis de identificar quando instalados numa
rede,  uma vez que o car�cter aut�nomo das RSSF, torna dif�cil distinguir um
comportamento errado de uma falha. Um sensor malicioso pode respeitar o
protocolo da rede, no entanto podem actuar de forma incorrecta levando a rede a
criar topologias especificas para o ataque (por exemplo, criando parti��es) ou
fazendo, por exemplo, toda a informa��o passar pelos n�s maliciosos, suprimindo
ou violando a informa��o. No que se refere aos ataques direccionados ao
encaminhamento, por serem parte do objectivo do estudo deste trabalho,
encontram-se definidos na pr�xima sec��o e s�o essencialmente instanciados pela
participa��o colaborativa ou isolada de n�s introduzidos  com o intuito de
afectar o normal funcionamento da rede.
\paragraph{Modelo bizantino: advers�rios bizantinos}
O modelo de ataques por intrus�o tem algumas parecen�as com as denominadas falhas
bizantinas\cite{Falhas_bizantinas}, s�o caracterizadas pela falhas arbitr�rias para as quais um
sistema n�o est�, � partida, preparada para lidar e que se pode traduzir em comportamentos
inesperados do sistema. Transpondo esta realidade para as RSSF, � dificil detectar a introdu��o de
n�s maliciosos, autonomos ou replicados a partir de um de um n� que ficou comprometido. No entanto
alguns autores \cite{PARNO_REPLICATION} t�m-se debru�ado sobre esta problem�tica a fim de dotarem os
algoritmos de mecanismos que permitam detectar a replica��o de n�s maliciosos numa RSSF.
Para se lidar com os ataques com comportamentos bizantinos, implementam-se mecanismos
probabilisticos que ainda que n�o possam mitigar o ataque por completo aumentam a resili�ncia e
acabam por transformar um ataque num mal menor, definindo at� onde pode ser comprometida a rede, ou
seja qual o n�mero de n�s que poder�o estar comprometidos mas que apesar disso a rede continua a
garantir a fiabilidade necess�ria para a opera��o.

 \subparagraph{Sum�rio}
    Mediante as vulnerabilidades de uma RSSF, � necess�rio estabelecer um modelo
de advers�rio com vista a poder mapear as capacidades e tipologias de ataques
deste em mecanismos de seguran�a com o prop�sito de lhes poder resistir ou
mitiga-los. O modelo de Dolev-Yao � o modelo de facto quando se trata da an�lise
de amea�as a redes , em que o meio de comunica��o est� sobre controlo do
advers�rio. No entanto, tratando-se de RSSF, este modelo per si n�o se vislumbra
suficiente para abarcar todas as problem�ticas de seguran�a a que este tipo de
redes est� sujeita. Surge assim a necessidade de, face � inseguran�a que cada n�
da rede representa, estender este modelo acrescentando-lhe um modelo de
intrus�o.
    Perante a exposi��o das RSSF, os ataques que se podem desencadear podem ser
diferentes dos observados nas redes convencionais sendo assim necess�rio
considerar outras tipologias de ataques. Assim, podemos classificar os ataques
como activos e passivos \cite{Stallings2005} e os atacantes como internos e
externos\cite{Karlof2003}.
Nestes ultimos ainda se pode classificar quanto aos recursos usados como
\textit{sensor-class} ou \textit{laptop-class}\cite{Karlof2003}. Os ataques que
se consideram para o estudo e
relacionados com as RSSF s�o: falsa informa��o de encaminhamento,
\textit{blackhole},\textit{sinkhole}, \textit{wormhole} e  \textit{sybil
attack}\cite{Douceur2002}.

 
\subsection{Ataques ao Encaminhamento}
Apesar de existirem ataques que podem ser dirigidos a qualquer das camadas da pilha da RSSF, em
particular apresentam-se os ataques relacionados com a camada de rede, respons�vel pelo
encaminhamento de dados. Os protocolo de encaminhamento em MANETs\cite{Corson1999} e em redes de
sensores, de uma forma geral, decomp�e-se em tr�s fases: descoberta dos caminhos, selec��o dos
caminhos e manuten��o da comunica��o pelos caminhos seleccionados. Os ataques a um algoritmo de
encaminhamento, normalmente, podem explorar as vulnerabilidades de cada uma destas fases de forma
especifica. Da�, em seguida se proceder � associa��o dos ataques espec�ficos a cada fase
apresentando as contramedidas que permitem mitig�-los.
\subsection{Ataques � organiza��o da rede e descoberta de n�s} \label{sect:subsec_ataq_org_rede}
Ap�s a descoberdos n�s vizinhos � necess�rio recolher informa��o para a constru��o das tabelas
de encaminhamento, isto nos protocolos do tipo \textit{table-driven}\cite{al-karaki_routing_2004}.
No entanto, em protocolos do tipo \textit{on-demand}\cite{al-karaki_routing_2004} esta fase �
desencadeada em cada in�cio de transmiss�o.
\paragraph*{\textbf{Falsifica��o de Informa��o de Encaminhamento}}
Este ataque tem impacto na forma��o da rede e na descoberta dos n�s. Induz a cria��o de entradas
incorrectas nas tabelas de encaminhamento, podendo tamb�m fazer com que estas fiquem lotadas e
inv�lidas. Nos protocolos \textit{on-demand}, o impacto pode ser menor, uma vez que obriga o 
atacante a injectar informa��o errada a cada ciclo de transmiss�o.
\paragraph*{\textbf{\textit{Rushing Attacks}}}
Outro ataque nesta fase � o \textit{rushing attack}\cite{Rushing_attacks_perrig} que � definido pela
explora��o, por parte do atacante, de uma janela de oportunidade para responder a um pedido de
caminho para um destino. Este ataque � efectivo quando o protocolo permite aceitar a primeira
resposta que recebe (ex: AODV\cite{Perkins1999}). Explorando isto, o atacante � sempre um
candidato a ser o pr�ximo encaminhador, uma vez que n�o respeita temporizadores nem condi��es de
resposta, podendo influenciar as rotas.
\subsubsection{Contra-medidas}
A aplica��o de mecanismos de autentica��o no protocolo de encaminhamento faz com que ataques de
falsifica��o de informa��o seja minimizados. Os n�s da rede podem partilhar chaves sim�tricas como
forma de autenticar as mensagens de dados e controlo do encaminhamento (RREQ e RREP). Desta forma, o
atacante como n�o possui a chaves necess�rias para a comunica��o, n�o poder� participar no
protocolo.

Para fazer face a ataques de \textit{rushing}, alguns autores \cite{Rushing_attacks_perrig}
apresentam dois mecanismos de defesa: reenvio aleat�rio de RREQ ()\textit{randomized RREQ
forwarding}) e detec��o segura (\textit{secure detection}). No primeiro mecanismo, cada n� guarda um
conjunto de mensagens RREQ escolhendo depois aleat�riamente um para reenviar. Ainda assim, pode ser
seleccionada uma mensagem RREQ maliciosa, da� a exist�ncia do segundo mecanismo, que proporciona a
troca de mensagens autenticadas entre dois n�s garantindo que as mensagens pertencem a n�s
leg�timos. 
\subsection{Ataques ao estabelecimento de rotas} \label{sect:subsec_ataq_est_rotas}
\begin{description}\addtolength{\itemsep}{-.50\baselineskip}
 \item[\textit{HELLO Flooding }]
Este ataque foi identificado primeiramente por \cite{Karlof2003}  sendo
definido como um ataque que explora alguns protocolos  que se fazem anunciar
aos seus vizinhos pela emiss�o de mensagens de \textit{HELLO}, informando-os da sua
proximidade presen�a\cite{Survey_wsn_Sec_issues}.
Os protocolos que assentam em localiza��o podem ser vulner�veis a este ataque,
uma vez que com um dispositivo do tipo \textit{laptop-class}\cite{Karlof2003}, usando um alcance
r�dio que cubra toda a rede, pode-se anunciar a todos os n�s como vizinho for�ando a
informa��o fluir atrav�s dele.
 \item[Ataque \textit{Sinkhole}]
Nas RSSF um dos modos de comunica��o � de um-para-muitos(\textit{one-to-many}).Este tipo
comunica��o  apresenta alguma vulnerabilidades a ataques do tipo
\textit{sinkhole}\cite{Sinkhole_attack}. Este ataque corresponde
a um atacante informar os n�s vizinhos de dados errados de encaminhamento anunciando-se como um n�
que tem boa comunica��o com o n� sink, tornando-se assim um ponto de passagem de informa��o. O
ataque � realizado enviando pacotes de RREQ, alterando a origem e aumentando o n�mero de
sequ�ncia como forma de fazer garantir que a informa��o se sobrep�e a qualquer outra, legitima, da
rede. Em  determinada altura, um atacante ter� a passar por ele um n�mero elevado de rotas, podendo
alterar ou encaminhar a informa��o de forma selectiva para outros destinos. Os ataques
\textit{table-driven} s�o vulner�veis a estes ataques  enquanto os protocolos baseados em
localiza��o n�o s�o devido �s suas rotas serem \textit{on-demand}.
\cite{Karlof2003,Survey_wsn_Sec_issues,Attaks_defenses_sec_in_wsn}
 \item[Ataque \textit{Wormhole}]
Neste tipo de ataque, apresentado por Perrig et al \cite{Wormhole_perrig} a colabora��o de dois n�s
maliciosos (normalmente a muitos hops de dist�ncia), quer sejam
n�s de \textit{laptop-class}\cite{Karlof2003} ou \textit{sensor-class}\cite{Karlof2003} , contribuem
para uma maior efectividade da ac��o de ataque. Assim, os atacantes estabelecem uma liga��o (ou
t�nel, normalmente de melhor qualidade - maior largura de banda) para comunicarem entre si. Um n�
malicioso captura pacotes ou partes de pacotes e envia-os pela liga��o privada para o outro atacante
para outro extremo da rede.
Este ataque � particularmente eficaz em redes ad-hoc e redes baseadas em localiza��o e sendo estas
compremetidas, n�o conseguiram estabelecer caminhos maiores do que dois hops causando interrup��es
nas comunica��es\cite{perrig_survey_ad_hoc,Survey_wsn_Sec_issues}.
Este ataque transforma o caminho os atacantes em n�s muito solicitados, pois apresentam-se aos
outros n�s participantes como tendo melhor liga��o e a menos dist�ncia do destino.
\cite{WAP_Wormhole}
\item[Ataque \textit{Sybil}]
Este ataque foi definido como um ataque que permitia atingir os mecanismos de redund�ncia
em armazenamento distribu�do em ambientes de ponto-a-ponto (peer-to-peer)\cite{Douceur2002}. Outra
defini��o que surge, agora associada �s RSSF, � a que o define como ``um dispositivo malicioso que
ilegitimamente assume m�ltiplas entidades''\cite{sybil_perrig}. Com estas defini��es e devido � sua
taxonomia � um ataque bastante efectivo contra protocolos de encaminhamento\cite{Karlof2003}. Em
particular dos protocolos que adoptam m�ltiplos caminhos, observa-se ent�o, que um n� ao assumir
v�rias identidades possibilita que na realidade os dados possam estar a passar por um mesmo n�
malicioso\cite{Survey_wsn_Sec_issues,Attaks_defenses_sec_in_wsn}.
\end{description}
\subsubsection{Contra-medidas}
Uma das formas de prevenir um ataque HELLO flooding\cite{Karlof1999} � a implementa��o de
mecanismos de respostas(\textit{aknowlege}) a an�ncios HELLO. Desta forma, caso o atacante esteja a
usar um meio de comunica��o potente, que cubra toda a rede, um n�, em que o atacante se encontre
fora do seu alcance, n�o aceitar� a an�ncio como v�lido.  Para al�m deste mecanismo � poss�vel
proceder � autentica��o da mensagem, certificando-a numa entidade central, que ao detectar que um
n� se anuncia como vizinho de muitos outros n�s, toma precau��es suspeitando que se trata de um
atacante podendo repudiar o n� emitindo uma mensagem para toda a rede\cite{Survey_wsn_Sec_issues}.

Alguns autores t�m vindo a desenvolver algoritmos que visam a detec��o de atacantes que desencadeam
ataques do tipo \textit{Sinkhole}\cite{Sinkhole_attack}, um desses mecanismos � o \textit{Sinkhole
Intrusion Detection Sistem} (SIDS)\cite{Sinkhole_attack} orientado para a detec��o destes ataques
ao protocolo DSR\cite{DSR}. Estes sistema prop�e tr�s mecanismos para detectar um atacante: i)
Discontinuidade de n�meros de sequ�ncia,  tendo em conta que um atacante tentar� usar n�meros de
sequ�ncia muito grandes, por forma a poder fazer prevalecer a sua inform��o, assim um n� pode
identificar os que crescem r�pidamente ou que n�o respeitam uma ordem crescente; ii)Taxa de pacotes
verificados, os vizinhos podem certificar a origem dos pacotes enviados por um n�, isto ser� dificil
de realizar em pacotes de atacantes, uma vez que eles alteram a origem, assim a rede poder� detectar
que est� sobre ataque se circularem muitos pacotes n�o certificados; iii)N�mero de caminhos a
passar por um n�, cada n� pode observar a sua tabela de encaminhamento e se detectar que existem
muitos caminhos a passar pelo mesmo n�, pode desconfiar estar sobre um ataque do tipo
\textit{Sinkhole}\cite{Survey_wsn_Sec_issues,Attaks_defenses_sec_in_wsn}

Alguns autores apresentam mecanismos como a utiliza��o de \textit{packet leashes}
\cite{packet_leashes_perrig} por forma a mitigar o ataque \textit{wormhole}. Preconizam que existem
dois tipos de condi��es para se aceitar os pacotes vindos de uma origem: baseado na localiza��o e
notempo. Assim, o primeiro permite um n� receptor, conhecendo a localiza��o da origem, saber se um 
pacote que atravessou a rede por um \textit{wormhole} calculando a distancia entre os dois pontos.
No segundo caso, baseia-se essencialmente no tempo de transmiss�o do pacote, exigindo ent�o a
sincroniza��o de rel�gios, se for muito r�pido a chegar ao destino, este n� assume que se est�
perante um ataque de \textit{wormhole}.

Para o ataque \textit{sybil} em \cite{sybil_perrig,Survey_wsn_Sec_issues}, s�o fornecidos dois
esquemas de protec��o:
\textit{radio resource testing} (cada vizinho s� pode transmitir num canal, selecciona uma canal
para ouvir, e envia uma mensagem, os n�s que n�o responderem s�o tratados como falsos) e
\textit{random key distribution}.(usando um modelo de \textit{key-pool} s�o atribuidas n keys de
um conjunto de m se dois n�s partilharem q key ent�o podem comunicar de forma segura, existe ainda
uma fun��o de hash, com base no ID do n� para gerar chaves, evitando que um n� possa ter multiplas
chaves)
\subsection{Ataques � manuten��o de rotas} \label{sect:subsec_ataq_manut_rotas}
\paragraph*{\textbf{Ataque \textit{Blackhole}}}
No ataque \textit{blackhole}\cite{HongmeiDeng2002} o atacante intercepta os pacotes destinados ao
n�/�rea que pretende comprometer, informando a origem que este se trata de um caminho de melhor
qualidade. Assim, for�a todo o tr�fego, dirigido ao destino alvo do ataque,  a circular atrav�s
dele. Por exemplo, no protocolo AODV\cite{Perkins1999}, por ser \textit{on-demand} permite que, na
fase de descoberta de uma rota, qualquer n�, que possua um caminho (suficiente recente), responda a
uma mensagem de RREQ. Com isto, este algoritmo de encaminhamento pode ficar sujeito a um ataque
de \textit{blackhole}, pois um n� malicioso interm�dio, pode responder com um caminho melhor,
apesar de n�o ter sequer caminho para o destino, originando um ``buraco negro'', interrompendo o
processo de comunica��o\cite{Survey_wsn_Sec_issues,Attaks_defenses_sec_in_wsn}.
\subsubsection{Contra-medidas}
Para mitigar os ataques de \textit{blackhole} existem v�rias propostas
\cite{blackhole_adhoc,Attaks_defenses_sec_in_wsn,HongmeiDeng2002} das quais se destacam as
que implementam os seguintes mecanismos: i) Confirma��o do destino, � enviada uma mensagens ACK por
cada pacote recibo pelo destino, pelo caminho inverso; iii) Defini��o de limites
de tempo para receber as mensagens de ACK. por parte do destino, ou ao inv�s, receber mensagens de
falha dos n�s interm�dios; iii) Mensagens de falha, quando num n� interm�dio detecta o fim do
temporizador de ACK, este gera uma mensagem de falha; iv) Caminho definido pela origem, ou seja, em
cada pacote � indicado, na origem, o caminho que deve ser seguido pelo pacote at� ao destino.
%\subsection{Ataques � reorganiza��o da rede }
\label{sect:subsec_ataq_reorg_rede}

