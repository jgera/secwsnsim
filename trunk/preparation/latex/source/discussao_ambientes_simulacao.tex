\subsection{Discuss�o}
A necessidade de recorrer a ambientes de simula��o para desenvolvimento de
tecnologias de RSSF � uma realidade dif�cil de contornar dadas as
caracter�sticas das RSSF e a necessidade de estudar todas as suas
caracteristicas. O que se tem vindo a observar � que, apesar das multiplicidade
de simuladores existentes\cite{SITE_DE_COMPARACAO_DE_SIM}, todos os ambientes
foram criados ou adaptados com caracteristicas muito ligadas para o fim a que se
proponham testar ou avaliar nas RSSF. Analisando algumas destas ferramentas
nota-se que n�o existe uma suficiente gen�rica e fl�xivel que permita avaliar
todos os aspectos de uma RSSF. No caso do Freemote, embora especificamente
desenvolvida para as RSSF, n�o incorpora modelos importantes como � o caso da
energia, factor altamente restritivo nestas redes, e que, devido ao impacto que
tem em cada componente, merece ser modelado e dado alguma aten��o. Ainda neste
simulador notem-se as car�ncias ao n�vel do modelo r�dio.

Uma das caracteristicas dos simuladores estudados, exceptuando o Freemote, � o
facto de serem orientados para redes sem fios em geral ou redes
\textit{ad-hoc}. Devido �s diferen�as que existem entre estas redes e as RSSF
torna-se dif�cil a avalia��o devida de protocolos ou aplica��es. No entanto �
de real�ar que o ShoX se apresenta como uma plataforma com bastantes modelos de
redes \textit{ad-hoc}, alguns dos quais aplicados a RSSF, nomeadamente o modelo
de consumo de energia e o modelo de r�dio(ainda que a norma implementada seja
IEEE802.11). Quanto � plataforma J-Sim esta tem de ser usada juntamente com um
componente de RSSF e requer a implementa��o dos modelos numa linguagem de
\textit{script } que implica: a aprendizagem de uma nova linguagem e o
\textit{overhead} de se ligar duas linguagens, que podem penalizar o desempenho
da simula��o. Por fim, a simplicidade do JProwler pode servir de ponto de
partida para a elabora��o de um simulador para teste e avalia��o de protocolos
de encaminhamento seguro, uma vez que tem um comportamento baseado no TinyOS e
com modelos do \textit{Mica2} pode ser estendido a incorporar m�dulos de gest�o
de energia e de an�lise gr�fica de comportamentos ou eventos que sirvam de
indicadores como por exemplo: energia consumida, fiabilidade, tempo de vida da
rede e cobertura