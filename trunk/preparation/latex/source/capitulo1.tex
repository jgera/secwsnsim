\chapter{Introdu��o} \label{cap:introducao}
\onehalfspacing
O texto da introdu��o, em fonte times-roman 12 e com um espa�amento e 
meio,  deve apresentar (i) uma extens�o ou introdu��o geral relativa ao resumo inicial, 
(ii) uma contextualizando o trabalho, apresentando as suas motiva��es , (iii) uma 
descri��o clara do problema ou foco do trabalho e terminando com (iv) a aproxima��o 
preconizada para a solu��o do problema ou do tratamento do tema focado, onde estejam 
claras as contribui��es previstas. Os alunos podem optar por apresentar esta introdu��o 
endere�ando os anteriores aspectos em sub-sec��es, como se exemplifica a seguir.

\section{Introdu��o geral ou Motiva��o} \label{sect:introducao}
A introdu��o, escrita com fonte times-roman-12, pode ter, como refer�ncia indicativa, 
entre 6 e 10 p�ginas, usando-se um espa�amento e meio.
\section{Descri��o e contexto (ou descri��o do problema)} \label{sect:descricao}
\section{Solu��o apresentada (ou �mbito do trabalho)} \label{sect:solucao}
\section{Principais contribui��es previstas} \label{sect:contribuicoes}

 As principais contribui��es previstas devem poder ser descritas em n�o mais do que uma p�gina, podendo adoptar-se, 
por exemplo, um estilo de apresenta��o por itens, com uma pequena descri��o de um 
par�grafo associado a cada item.


