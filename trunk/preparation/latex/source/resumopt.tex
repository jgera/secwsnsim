\resumo
As redes de sensores s�o uma tecnologia emergente no dom�nio da monitoriza��o
, de forma aut�noma, de ambientes fisicos. S�o formadas por pequenos
dispositivos que se auto-organizam por modo a cobrirem uma �rea geogr�fica. Esta
autonomia e auto-organiza��o apresenta alguns desafios relacionados com os
aspectos de seguran�a, nomeadamente, no que concerne com o encaminhamento de
dados. Assim, o trabalho a realizar pretende contribuir para a cria��o
de um modelo sist�mico para o estudo de protocolos de encaminhamento seguro em
redes de sensores sem fios (RSSF). A defini��o do modelo de advers�rio � o passo
inicial para o enquadramento das tipologias de ataque que se pretende avaliar.
Aliado ao modelo formal de Dolev-Yao, orientado para os ataques ao meio de
comunica��o, o estudo de novos modelos de advers�rio, relacionados com a
intrus�o ou captura de n�s, quer sejam bizantinos ou probabilisticos, �
pertinente e apresentado dentro do �mbito deste trabalho.

Com vista a tornar as RSSF resistentes a algumas tipologias de ataques
preconizadas no modelo de advers�rio, t�m vindo a ser desenvolvidos diversos
algoritmos de encaminhamento seguro. Pretende-se estudar alguns destes
algoritmos, representantes do estado da arte neste dom�nio, estabelecendo uma
matriz de medidas de resist�ncia ao modelo de advers�rio. Os algoritmos alvo
deste estudo s�o o SIGF, Clean-slate e INSENS, entendendo que no seu todo cobrem
todas tipologias de ataque em an�lise, no entanto cada um tem lacunas no seu
desenho quando direccionados alguns destes ataques. Ainda como contributo deste
trabalho, pretende-se modelar um ambiente de simula��o que permita
avaliar/analisar, no quadro de ataques definido, as caracteristicas dos
protocolos de encaminhamento em RSSF em m�teria de seguran�a e resist�ncia a
ataques. Portanto, importa estudar e estabelecer crit�rios para an�lise de
sistemas de simula��o para RSSF das as caracteristicas conhecidas, por exemplo
os recursos limitados dos modelos de energia, processamento e comunica��o.
% Palavras-chave do resumo em Portugu�s
\begin{keywords}
Redes de sensores sem fios,Protocolos de encaminhamento seguros,Simula��o
de redes de sensores, Ataque por intrus�o
\end{keywords}
% to add an extra black line
~\\ ~\\\rule{\textwidth}{0.2mm}


