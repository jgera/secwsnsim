\begin{table}[H]
\centering
\begin{scriptsize}
\begin{tabular}[t]{|l|l|}\hline
\mc{1}{|c}{\textbf{M�todo}} & \mc{1}{|c|}{\textbf{Descri��o}}\\\hline\hline
\textit{sendMessage(m,app)} & Recep��o de uma mensagem vinda da camada
aplica��o\\\hline
\textit{onSendMessage(m)} & Envio de uma mensagem para o n�vel MAC\\\hline
\textit{onReceiveMessage(m)} & Ac��es na recep��o de uma mensagem (ex. avalia��o
do tipo da mensagem)\\\hline
\textit{onRouteMessage(m)} & Ac��es de encaminhamento de uma mensagem\\\hline
\textit{onStartUp()} & Ac��es de inicializa��o do protocolo de
encaminhamento\\\hline
\textit{setupAttacks()} & Cria��o das instancias dos ataques que podem ser
desencadeados\\\hline
\textit{initAttacks()} & Inicializa��o de dados referentes a ataques ao
encaminhamento definidos\\\hline
\textit{sendMessageToAir()} & Especifica as ac��es de enviar a mensagem para o
meio\\\hline
\textit{sendMessageDone()} & Assinala o envio de uma mensagem com
sucesso\\\hline
\textit{newRound()} & Permite activar uma ac��o em protocolos que podem ter
novas rondas\\\hline
\textit{encapsulateMessage(m)} & Cria um "envelope" para a mensagem de enviada
pela aplica��o\\\hline
\textit{onStable()} & Permite desencadear uma ac��o assim que o n� entrar em
modo de estabiliza��o\\\hline
\textit{send(m)} & Este m�todo dever� ser sempre usado para enviar as
mensagens\\
& pois � o m�todo condutor de todo o processo de transmiss�o/ataque/controlo dos
dados\\\hline
\end{tabular}
\caption{API ao n�vel do \textit{RoutingLayer} \label{tab:APIRouting}
\end{scriptsize}
\end{table}