\begin{table}[H]
 \centering
\begin{footnotesize}
\begin{tabular}[t]{|l|p{11cm}|}\hline
\textbf{Opera��o} & \textbf{Descri��o} \\\hline
\textbf{rebroadcastRREQMessage()} &se foi a primeira vez que foi recebida uma
mensagem de Route Request ent�o � re-enviada para os vizinhos adicionando a
informa��o do n� actual, � iniciado o temporizador para envio de
\textit{feedback} ap�s se processada a informa��o;\\\hline
\textbf{sendFeedbackMessageInfo()} & ap�s finaliza��o do temporizador � enviada
para a esta��o base a informa��o relativa aos vizinhos conhecidos e as
respectivas assinaturas recebidas;\\\hline
\textbf{processFDBKMessage()} & uma vez que as mensagens de feedback circulam em
sentido contr�rio ao das do tipo Route Request, um qualquer n� interm�dio tem
que fazer circular no sentido ascendente as mensagens recebidas dos n�s
que receberam a informa��o de request enviada pelo sensor actual;\\\hline
\textbf{updateRoutingStatus()} & quando � recebida a informa��o da esta��o base
com a tabela de encaminhamento cada sensor actualiza a informa��o das tabelas e
passa e muda o seu estado para est�vel, servindo isto de indicador de que � um
n� encaminhador, ou seja dotado de uma tabela de encaminhamento v�lida.\\\hline
\end{tabular}
\caption{Tabela de opera��es princ�pais da esta��o base}
\label{tab:operacoes_sensores}
\end{footnotesize}
\end{table}
