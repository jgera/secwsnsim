\subsection{Ataques � manuten��o de rotas} \label{sect:subsec_ataq_manut_rotas}
\begin{description}\addtolength{\itemsep}{-.50\baselineskip}
 \item[Ataque \textit{Blackhole}]
Entende-se por um ataque do tipo \textit{blackhole}\cite{HongmeiDeng2002} um ataque desencadeado
por um n� malicioso que intercepta os pacotes dirigidos ao local
que pretende atacar e informa a origem como sendo um caminho de melhor qualidade. Assim, for�a todo
o tr�fego, dirigido ao destino, alvo do ataque,  a circular atrav�s dele. Por exemplo protocolo AODV
\cite{Perkins1999} tratando-se de um protocolo on-demand permite que, na fase de
descoberta de uma rota, qualquer n�, que possua um caminho (suficiente recente), responda a uma
mensagem de RREQ. Com isto, este algoritmo de encaminhamento fica pode ficar sujeito a um ataque de
blackhole, pois um n� intermedi�rio, malicioso, pode responder com um caminho melhor apesar de n�o
ter sequer caminho para o destino originando assim um ``buraco negro'', desta forma interrompe o
processo de comunica��o\cite{Survey_wsn_Sec_issues,Attaks_defenses_sec_in_wsn}.
\end{description}
\subsubsection{Contra-medidas}
Para mitigar os ataques de \textit{blackhole} existem v�rias propostas
\cite{blackhole_adhoc,Attaks_defenses_sec_in_wsn,HongmeiDeng2002} das quais se destacam as
que
implementam os seguintes mecanismos: i) Confirma��o do Destino, � enviada uma mensagens ACK por
cada pacote recibo pelo destino, pelo caminho inverso;iii) \textit{Timeouts}: S�o definidos limites
de tempo, entre os n�s interm�dios e a origem, para receber as mensagens de ACK por parte do
destino, ou ao inv�s, receber mensagens de falhas; iii)Mensagens de falha, quando num n�
interm�dio expira um temporizador de ACK, este gera uma mensagem de falha; iv) Caminho definido
pela origem, em cada pacote � indicado pela origem o caminho que deve ser seguido pelo pacote at�
ao destino.
