\subsection{Ataques � manuten��o de rotas} \label{sect:subsec_ataq_manut_rotas}
\begin{description}
 \item[Ataque \textit{Blackhole}]
    Entende-se por um ataque do tipo blackhole [Hongmei Deng; Wei Li; Agrawal, D.P., ``Routing
security in wireless ad hoc networks,'' Communications Magazine, IEEE , vol.40, no.10, pp. 70-75,
Oct 2002] um ataque desencadeado por um n� malicioso que intercepta os pacotes dirigidos ao local
que pretende atacar e informa a origem como sendo um caminho de melhor qualidade. Assim, for�a todo
o tr�fego, dirigido ao destino, alvo do ataque,  a circular atrav�s dele. Por exemplo protocolo AODV
[AODV REF] tratando-se de um protocolo on-demand permite que, na fase de descoberta de uma rota,
qualquer n�, que possua um caminho (suficiente recente), responda a uma mensagem de RREQ. Com isto,
este algoritmo de encaminhamento fica pode ficar sujeito a um ataque de blackhole, pois um n�
intermedi�rio, malicioso, pode responder com um caminho melhor apesar de n�o ter sequer caminho para
o destino originando assim um ``buraco negro'', desta forma interrompe o processo de comunica��o.
 \item[Ataque \textit{Spam}]
 \end{description}
\subsubsection{Contra-medidas}