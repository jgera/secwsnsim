\section{Modelo de Advers�rio e Servi�os de
Seguran�a} \label{sect:sec_mod_adversario_serv_seg}
Quando se tratam de quest�es de seguran�a, qualquer se seja o seu dom�nio,
existe uma primeira pergunta que cumpre fazer: ``quais s�o as amea�as/ataques a
que est� sujeito o objecto que se pretende manter seguro?''. Esta pergunta
possibilita, desde logo, encetar uma caminhada que visa a identifica��o de quais
os poss�veis atacantes, que capacidades estes possuem, quais os meios e modos
que estes podem utilizar e em que momento o ataque se pode desencadear. Esta
abordagem, um tanto ou quanto generalista, � suficiente para ilustrar a forma
como se pretende orientar o estudo e com isto apresentar, nas mesmas vertentes,
o modelo de advers�rio que enforma este trabalho.
 \subsubsection{Modelo de Dolev-Yao}\label{sect:subsec_dolev_yao}
Um dos modelos de advers�rio mais conhecidos, quando se trata de
an�lise formal de protocolos seguros, � o modelo de Dolev-Yao \cite{Dolev1983}. 
Assim, neste modelo, � considerado que a rede est� sobre o dom�nio do
advers�rio, que perante este facto pode extrair, reordenar, reenviar, alterar e
apagar as mensagens que circulam entre quaisquer dois principais legitimos. Com
esta assump��o, entende-se portanto, que o advers�rio transporta a mensagem e
com isso adopta um ataque do tipo \textit{man-in-the-middle}\cite{Stall2005},
com
comportamento incorrecto, que o leva a poder alterar o destinat�rio, atribuir
uma falsa origem, analisar o tr�fego ou alterar as mensagens. Este
funcionamento, entenda-se, n�o � comparado � intrus�o mas sim � intercep��o de
mensagens que pode ser mitigado usando mecanismos de criptografia.

As tipologias de ataque, consideradas pelo o modelo de advers�rio de Dolev-Yao
s�o  instanciadas pela norma X800 \cite{ITU-T1991} que pretende
normalizar
uma arquitectura de seguran�a para o modelo OSI, oferecendo uma abordagem
sistem�tica para o desenho de sistemas seguros. Esta norma considera a seguran�a
sobre tr�s aspectos: ataque, mecanismo e servi�o de
seguran�a\cite{Stall2005}. O
primeiro refere-se � forma usada para comprometer um sistema, por exemplo,
alterando ou tendo acesso n�o autorizado autorizado a dados desse sistema. Na
literatura, algumas vezes usam-se os temos ataque e amea�a para denominarem o
mesmo efeito, no entanto recorrendo ao RFC 2828 \cite{RFC2828} podemos
definir amea�a como uma potencial viola��o de seguran�a, ou seja � apenas uma
possibilidade que pode ser usada para desencadear um ataque explorando uma
vulnerabilidade; no caso do ataque, trata-se da explora��o inteligente de uma ou
mais amea�as que resultam na viola��o com sucesso de um sistema que se pretendia
seguro. O segundo aspecto considerado, na norma X.800, s�o os mecanismos de
seguran�a, que se entende como o processo que permite detectar, prevenir ou
recuperar de uma ataque � seguran�a (ex: encripta��o, controlo de acesso,
assinatura digital)\cite{Stall2005}. Por fim, o terceiro aspecto define os
servi�os
que, fazendo uso de um ou mais mecanismos de seguran�a, permitem resistir a
ataques dirigidos a determinada fonte de informa��o, quer seja durante o
processamento ou durante a comunica��o.
 \subsubsection{Modelo de Intrus�o em RSSF}\label{sect:subsec_intrusao}
Considerando o estudo de seguran�a numa RSSF, e dada a sua exposi��o
natural, nomeadamente a f�sica, colocando cada sensor ao alcance de um qualquer
advers�rio, torna relevante a considera��o de novos modelos de ataque.
Considerando que cada rede pode ser constitu�da por milhares de sensores, cada
um deles � um ponto de ataque, na impossibilidade de se proteger ou monitorizar
todos os sensores instalados\cite{Perriga}. Assim
as RSSF v�m-se sujeitas a um modelo de advers�rio que difere das redes com/sem
fios convencionais. Um advers�rio pode estar perto da rede e ter acesso aos
sensores e com isto ``roubar'' um ou parte dos sensores da rede com vista a
explorar os segredos ou material criptogr�fico usados para a comunica��o.
Podemos ent�o tipificar estes ataques como sendo por intrus�o. 
Este tipo de ataques podem ser definidos por ataques desde o n�vel
MAC\cite{Xiao2006} at� ao n�vel de intrus�o f�sica  em que um actor externo,
tendo acesso a um ou m�s sensores leg�timos, descobre os segredos criptogr�ficos
permitindo-lhe replicar\cite{Parno2005} os segredos para sensores maliciosos,
que depois de introduzidos podem agir de forma coordenada comprometendo a rede.
Conseguida a intrus�o, o atacante pode induzir nos sensores leg�timos
comportamentos incorrectos baseados na informa��o falsa introduzida pelos
sensores maliciosos, influenciando o processo de encaminhamento (denominados de
ataques ao encaminhamento).  Note-se, por exemplo, que estes ataques t�m
caracter�sticas que os tornam dif�ceis de identificar quando instalados numa
rede,  uma vez que o car�cter aut�nomo das RSSF, torna dif�cil distinguir um
comportamento errado de uma falha. Um sensor malicioso pode respeitar o
protocolo da rede, no entanto podem actuar de forma incorrecta levando a rede a
criar topologias especificas para o ataque (por exemplo, criando parti��es) ou
fazendo, por exemplo, toda a informa��o passar pelos n�s maliciosos, suprimindo
ou violando a informa��o. No que se refere aos ataques direccionados ao
encaminhamento, por serem parte do objectivo do estudo deste trabalho,
encontram-se definidos na pr�xima sec��o e s�o essencialmente instanciados pela
participa��o colaborativa ou isolada de n�s introduzidos  com o intuito de
afectar o normal funcionamento da rede.
\paragraph{Modelo bizantino: advers�rios bizantinos}
O modelo de ataques por intrus�o tem algumas parecen�as com as denominadas falhas
bizantinas\cite{Falhas_bizantinas}, s�o caracterizadas pela falhas arbitr�rias para as quais um
sistema n�o est�, � partida, preparada para lidar e que se pode traduzir em comportamentos
inesperados do sistema. Transpondo esta realidade para as RSSF, � dificil detectar a introdu��o de
n�s maliciosos, autonomos ou replicados a partir de um de um n� que ficou comprometido. No entanto
alguns autores \cite{PARNO_REPLICATION} t�m-se debru�ado sobre esta problem�tica a fim de dotarem os
algoritmos de mecanismos que permitam detectar a replica��o de n�s maliciosos numa RSSF.
Para se lidar com os ataques com comportamentos bizantinos, implementam-se mecanismos
probabilisticos que ainda que n�o possam mitigar o ataque por completo aumentam a resili�ncia e
acabam por transformar um ataque num mal menor, definindo at� onde pode ser comprometida a rede, ou
seja qual o n�mero de n�s que poder�o estar comprometidos mas que apesar disso a rede continua a
garantir a fiabilidade necess�ria para a opera��o.

 \subsection{Tipos de Ataques a RSSF}\label{sect:subsec_tipos_ataques}
Devido � arquitectura de \textit{software} das RSSF, estas est�o sujeitas a uma
serie de amea�as que desde logo podem propiciar ataques. 
 
\subsubsection{Ataques ao n�vel fisico}

\subsubsection{Ataques ao n�vel rede}

\subsubsection{Ataques ao n�vel aplica��o}
 \subparagraph{Sum�rio}
    Mediante as vulnerabilidades de uma RSSF, � necess�rio estabelecer um modelo
de advers�rio com vista a poder mapear as capacidades e tipologias de ataques
deste em mecanismos de seguran�a com o prop�sito de lhes poder resistir ou
mitiga-los. O modelo de Dolev-Yao � o modelo de facto quando se trata da an�lise
de amea�as a redes , em que o meio de comunica��o est� sobre controlo do
advers�rio. No entanto, tratando-se de RSSF, este modelo per si n�o se vislumbra
suficiente para abarcar todas as problem�ticas de seguran�a a que este tipo de
redes est� sujeita. Surge assim a necessidade de, face � inseguran�a que cada n�
da rede representa, estender este modelo acrescentando-lhe um modelo de
intrus�o.
    Perante a exposi��o das RSSF, os ataques que se podem desencadear podem ser
diferentes dos observados nas redes convencionais sendo assim necess�rio
considerar outras tipologias de ataques. Assim, podemos classificar os ataques
como activos e passivos \cite{Stallings2005} e os atacantes como internos e
externos\cite{Karlof2003}.
Nestes ultimos ainda se pode classificar quanto aos recursos usados como
\textit{sensor-class} ou \textit{laptop-class}\cite{Karlof2003}. Os ataques que
se consideram para o estudo e
relacionados com as RSSF s�o: falsa informa��o de encaminhamento,
\textit{blackhole},\textit{sinkhole}, \textit{wormhole} e  \textit{sybil
attack}\cite{Douceur2002}.

 