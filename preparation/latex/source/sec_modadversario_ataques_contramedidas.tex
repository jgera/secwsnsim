\section{Modelo de Advers�rio, Ataques ao Encaminhamento e
Contra-medidas} \label{sect:sec_mod_adversario_ataq_contramedidas}

S�o v�rios os ataques que se pode direccionar contra a pilha da arquitectura de
uma RSSF. Cada uma das camadas da pilha tem vulnerabilidades pr�prias das
fun��es que desempenha. No escopo deste trabalho o foco � a seguran�a ao n�vel
dos protocolos de encaminhamento, representada pela camada de rede, da�
nesta sec��o se apresentar com algum detalhe as fases tradicionais de um
protocolo de encaminhamento em MANETs\cite{Corson1999} e em particular em
RSSF.\\
Os protocolo de encaminhamento em redes de sensores, de uma forma geral
dividem-se em tr�s fases: descoberta dos caminhos, selec��o dos caminhos,
manuten��o da comunica��o pelos caminhos seleccionados. Importa neste momento
real�ar o facto de que os ataques a um algoritmo de encaminhamento normalmente
exploraram as vulnerabilidades de cada uma destas fases de forma especifica.
Da�, em seguida se proceder � associa��o dos ataques espec�ficos que se podem
desencadear em cada fase e como estes podem ser mitigados aplicando determinados
mecanismos de seguran�a como contra-medidas.
\subsection{Ataques � organiza��o da rede e descoberta de n�s} \label{sect:subsec_ataq_org_rede}
Ap�s a descoberdos n�s vizinhos � necess�rio recolher informa��o para a constru��o das tabelas
de encaminhamento, isto nos protocolos do tipo \textit{table-driven}\cite{al-karaki_routing_2004}.
No entanto, em protocolos do tipo \textit{on-demand}\cite{al-karaki_routing_2004} esta fase �
desencadeada em cada in�cio de transmiss�o.
\paragraph*{\textbf{Falsifica��o de Informa��o de Encaminhamento}}
Este ataque tem impacto na forma��o da rede e na descoberta dos n�s. Induz a cria��o de entradas
incorrectas nas tabelas de encaminhamento, podendo tamb�m fazer com que estas fiquem lotadas e
inv�lidas. Nos protocolos \textit{on-demand}, o impacto pode ser menor, uma vez que obriga o 
atacante a injectar informa��o errada a cada ciclo de transmiss�o.
\paragraph*{\textbf{\textit{Rushing Attacks}}}
Outro ataque nesta fase � o \textit{rushing attack}\cite{Rushing_attacks_perrig} que � definido pela
explora��o, por parte do atacante, de uma janela de oportunidade para responder a um pedido de
caminho para um destino. Este ataque � efectivo quando o protocolo permite aceitar a primeira
resposta que recebe (ex: AODV\cite{Perkins1999}). Explorando isto, o atacante � sempre um
candidato a ser o pr�ximo encaminhador, uma vez que n�o respeita temporizadores nem condi��es de
resposta, podendo influenciar as rotas.
\subsubsection{Contra-medidas}
A aplica��o de mecanismos de autentica��o no protocolo de encaminhamento faz com que ataques de
falsifica��o de informa��o seja minimizados. Os n�s da rede podem partilhar chaves sim�tricas como
forma de autenticar as mensagens de dados e controlo do encaminhamento (RREQ e RREP). Desta forma, o
atacante como n�o possui a chaves necess�rias para a comunica��o, n�o poder� participar no
protocolo.

Para fazer face a ataques de \textit{rushing}, alguns autores \cite{Rushing_attacks_perrig}
apresentam dois mecanismos de defesa: reenvio aleat�rio de RREQ ()\textit{randomized RREQ
forwarding}) e detec��o segura (\textit{secure detection}). No primeiro mecanismo, cada n� guarda um
conjunto de mensagens RREQ escolhendo depois aleat�riamente um para reenviar. Ainda assim, pode ser
seleccionada uma mensagem RREQ maliciosa, da� a exist�ncia do segundo mecanismo, que proporciona a
troca de mensagens autenticadas entre dois n�s garantindo que as mensagens pertencem a n�s
leg�timos. 
\subsection{Ataques ao estabelecimento de rotas} \label{sect:subsec_ataq_est_rotas}
\begin{description}\addtolength{\itemsep}{-.50\baselineskip}
 \item[\textit{HELLO Flooding }]
Este ataque foi identificado primeiramente por \cite{Karlof2003}  sendo
definido como um ataque que explora alguns protocolos  que se fazem anunciar
aos seus vizinhos pela emiss�o de mensagens de \textit{HELLO}, informando-os da sua
proximidade presen�a\cite{Survey_wsn_Sec_issues}.
Os protocolos que assentam em localiza��o podem ser vulner�veis a este ataque,
uma vez que com um dispositivo do tipo \textit{laptop-class}\cite{Karlof2003}, usando um alcance
r�dio que cubra toda a rede, pode-se anunciar a todos os n�s como vizinho for�ando a
informa��o fluir atrav�s dele.
 \item[Ataque \textit{Sinkhole}]
Nas RSSF um dos modos de comunica��o � de um-para-muitos(\textit{one-to-many}).Este tipo
comunica��o  apresenta alguma vulnerabilidades a ataques do tipo
\textit{sinkhole}\cite{Sinkhole_attack}. Este ataque corresponde
a um atacante informar os n�s vizinhos de dados errados de encaminhamento anunciando-se como um n�
que tem boa comunica��o com o n� sink, tornando-se assim um ponto de passagem de informa��o. O
ataque � realizado enviando pacotes de RREQ, alterando a origem e aumentando o n�mero de
sequ�ncia como forma de fazer garantir que a informa��o se sobrep�e a qualquer outra, legitima, da
rede. Em  determinada altura, um atacante ter� a passar por ele um n�mero elevado de rotas, podendo
alterar ou encaminhar a informa��o de forma selectiva para outros destinos. Os ataques
\textit{table-driven} s�o vulner�veis a estes ataques  enquanto os protocolos baseados em
localiza��o n�o s�o devido �s suas rotas serem \textit{on-demand}.
\cite{Karlof2003,Survey_wsn_Sec_issues,Attaks_defenses_sec_in_wsn}
 \item[Ataque \textit{Wormhole}]
Neste tipo de ataque, apresentado por Perrig et al \cite{Wormhole_perrig} a colabora��o de dois n�s
maliciosos (normalmente a muitos hops de dist�ncia), quer sejam
n�s de \textit{laptop-class}\cite{Karlof2003} ou \textit{sensor-class}\cite{Karlof2003} , contribuem
para uma maior efectividade da ac��o de ataque. Assim, os atacantes estabelecem uma liga��o (ou
t�nel, normalmente de melhor qualidade - maior largura de banda) para comunicarem entre si. Um n�
malicioso captura pacotes ou partes de pacotes e envia-os pela liga��o privada para o outro atacante
para outro extremo da rede.
Este ataque � particularmente eficaz em redes ad-hoc e redes baseadas em localiza��o e sendo estas
compremetidas, n�o conseguiram estabelecer caminhos maiores do que dois hops causando interrup��es
nas comunica��es\cite{perrig_survey_ad_hoc,Survey_wsn_Sec_issues}.
Este ataque transforma o caminho os atacantes em n�s muito solicitados, pois apresentam-se aos
outros n�s participantes como tendo melhor liga��o e a menos dist�ncia do destino.
\cite{WAP_Wormhole}
\item[Ataque \textit{Sybil}]
Este ataque foi definido como um ataque que permitia atingir os mecanismos de redund�ncia
em armazenamento distribu�do em ambientes de ponto-a-ponto (peer-to-peer)\cite{Douceur2002}. Outra
defini��o que surge, agora associada �s RSSF, � a que o define como ``um dispositivo malicioso que
ilegitimamente assume m�ltiplas entidades''\cite{sybil_perrig}. Com estas defini��es e devido � sua
taxonomia � um ataque bastante efectivo contra protocolos de encaminhamento\cite{Karlof2003}. Em
particular dos protocolos que adoptam m�ltiplos caminhos, observa-se ent�o, que um n� ao assumir
v�rias identidades possibilita que na realidade os dados possam estar a passar por um mesmo n�
malicioso\cite{Survey_wsn_Sec_issues,Attaks_defenses_sec_in_wsn}.
\end{description}
\subsubsection{Contra-medidas}
Uma das formas de prevenir um ataque HELLO flooding\cite{Karlof1999} � a implementa��o de
mecanismos de respostas(\textit{aknowlege}) a an�ncios HELLO. Desta forma, caso o atacante esteja a
usar um meio de comunica��o potente, que cubra toda a rede, um n�, em que o atacante se encontre
fora do seu alcance, n�o aceitar� a an�ncio como v�lido.  Para al�m deste mecanismo � poss�vel
proceder � autentica��o da mensagem, certificando-a numa entidade central, que ao detectar que um
n� se anuncia como vizinho de muitos outros n�s, toma precau��es suspeitando que se trata de um
atacante podendo repudiar o n� emitindo uma mensagem para toda a rede\cite{Survey_wsn_Sec_issues}.

Alguns autores t�m vindo a desenvolver algoritmos que visam a detec��o de atacantes que desencadeam
ataques do tipo \textit{Sinkhole}\cite{Sinkhole_attack}, um desses mecanismos � o \textit{Sinkhole
Intrusion Detection Sistem} (SIDS)\cite{Sinkhole_attack} orientado para a detec��o destes ataques
ao protocolo DSR\cite{DSR}. Estes sistema prop�e tr�s mecanismos para detectar um atacante: i)
Discontinuidade de n�meros de sequ�ncia,  tendo em conta que um atacante tentar� usar n�meros de
sequ�ncia muito grandes, por forma a poder fazer prevalecer a sua inform��o, assim um n� pode
identificar os que crescem r�pidamente ou que n�o respeitam uma ordem crescente; ii)Taxa de pacotes
verificados, os vizinhos podem certificar a origem dos pacotes enviados por um n�, isto ser� dificil
de realizar em pacotes de atacantes, uma vez que eles alteram a origem, assim a rede poder� detectar
que est� sobre ataque se circularem muitos pacotes n�o certificados; iii)N�mero de caminhos a
passar por um n�, cada n� pode observar a sua tabela de encaminhamento e se detectar que existem
muitos caminhos a passar pelo mesmo n�, pode desconfiar estar sobre um ataque do tipo
\textit{Sinkhole}\cite{Survey_wsn_Sec_issues,Attaks_defenses_sec_in_wsn}

Alguns autores apresentam mecanismos como a utiliza��o de \textit{packet leashes}
\cite{packet_leashes_perrig} por forma a mitigar o ataque \textit{wormhole}. Preconizam que existem
dois tipos de condi��es para se aceitar os pacotes vindos de uma origem: baseado na localiza��o e
notempo. Assim, o primeiro permite um n� receptor, conhecendo a localiza��o da origem, saber se um 
pacote que atravessou a rede por um \textit{wormhole} calculando a distancia entre os dois pontos.
No segundo caso, baseia-se essencialmente no tempo de transmiss�o do pacote, exigindo ent�o a
sincroniza��o de rel�gios, se for muito r�pido a chegar ao destino, este n� assume que se est�
perante um ataque de \textit{wormhole}.

Para o ataque \textit{sybil} em \cite{sybil_perrig,Survey_wsn_Sec_issues}, s�o fornecidos dois
esquemas de protec��o:
\textit{radio resource testing} (cada vizinho s� pode transmitir num canal, selecciona uma canal
para ouvir, e envia uma mensagem, os n�s que n�o responderem s�o tratados como falsos) e
\textit{random key distribution}.(usando um modelo de \textit{key-pool} s�o atribuidas n keys de
um conjunto de m se dois n�s partilharem q key ent�o podem comunicar de forma segura, existe ainda
uma fun��o de hash, com base no ID do n� para gerar chaves, evitando que um n� possa ter multiplas
chaves)
\subsection{Ataques � manuten��o de rotas} \label{sect:subsec_ataq_manut_rotas}
\paragraph*{\textbf{Ataque \textit{Blackhole}}}
No ataque \textit{blackhole}\cite{HongmeiDeng2002} o atacante intercepta os pacotes destinados ao
n�/�rea que pretende comprometer, informando a origem que este se trata de um caminho de melhor
qualidade. Assim, for�a todo o tr�fego, dirigido ao destino alvo do ataque,  a circular atrav�s
dele. Por exemplo, no protocolo AODV\cite{Perkins1999}, por ser \textit{on-demand} permite que, na
fase de descoberta de uma rota, qualquer n�, que possua um caminho (suficiente recente), responda a
uma mensagem de RREQ. Com isto, este algoritmo de encaminhamento pode ficar sujeito a um ataque
de \textit{blackhole}, pois um n� malicioso interm�dio, pode responder com um caminho melhor,
apesar de n�o ter sequer caminho para o destino, originando um ``buraco negro'', interrompendo o
processo de comunica��o\cite{Survey_wsn_Sec_issues,Attaks_defenses_sec_in_wsn}.
\subsubsection{Contra-medidas}
Para mitigar os ataques de \textit{blackhole} existem v�rias propostas
\cite{blackhole_adhoc,Attaks_defenses_sec_in_wsn,HongmeiDeng2002} das quais se destacam as
que implementam os seguintes mecanismos: i) Confirma��o do destino, � enviada uma mensagens ACK por
cada pacote recibo pelo destino, pelo caminho inverso; iii) Defini��o de limites
de tempo para receber as mensagens de ACK. por parte do destino, ou ao inv�s, receber mensagens de
falha dos n�s interm�dios; iii) Mensagens de falha, quando num n� interm�dio detecta o fim do
temporizador de ACK, este gera uma mensagem de falha; iv) Caminho definido pela origem, ou seja, em
cada pacote � indicado, na origem, o caminho que deve ser seguido pelo pacote at� ao destino.
\subsection{Ataques � reorganiza��o da rede }
\label{sect:subsec_ataq_reorg_rede}

