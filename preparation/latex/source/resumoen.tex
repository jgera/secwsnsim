\abstract
Sensor networks are an emerging technology in the field of monitoring  physical environments, in an
autonomous manner. They are formed by small, self-organized devices which cover a geographical area
and can form a large scale network with thousands of nodes. This autonomy and self-organization
present some security challenges, concerning data routing, in particular.

This work aims to contribute to the creation of a systemic model for the study of secure routing
protocols in wireless sensor networks (WSN). The definition of the opponent model is the first step
to an enhanced understanding of the different types of attack. Coupled with the formal Dolev-Yao
model, which focuses on the attacks on communication media, the study of new adversary models,
related to intrusion and capture of nodes, is relevant and presented within this work
In order to make the WSN resistant to some types of attacks, several secure routing algorithms have
been developed. The aim is to study some of these algorithms, representatives of the state of the
art in this field, establishing a matrix of resistance to each type of opponent, which then allows
the evaluation of their effectiveness.

The major contribution of this study is the design of an innovative simulation environment, since
some features to implement are not found in existing WSN simulation systems. It will provide the
opportunity to design and evaluate routing algorithms, designed to be secure, when subjected to the
attacks defined in the opponent model. This evaluation will primarily focuses on the analysis of
certain properties, such as energy consumption, reliability, latency, data accuracy and correction
of the protocol's behavior.

% Palavras-chave do resumo em Ingl�s
\begin{keywords}
Wireless Sensor Networks, Secure Routing Protocols, WSN Simulation, Intrusion
Atack
\end{keywords} 
% to add an extra black line
~\\ ~\\ \rule{\textwidth}{0.2mm}
