\section{ Discuss�o e Resumo do Trabalho Relacionado}
As redes de sensores sem fios representam um enorme desafio para a investiga��o de sistemas e
protocolos de seguran�a. As caracter�sticas que as tornam numa mais valia, para a opera��o em
ambientes remotos, apresentam-se como sendo as suas maiores vulnerabilidades em termos de
seguran�a. Este paradoxo � contornado com mecanismos de seguran�a inovadores e que se distinguem
dos existentes nas redes convencionais. Assim, passada em revista as diversas dimens�es que se
pretende abarcar na futura disserta��o:  protocolos de encaminhamento seguro em RSSF e
plataformas de simula��o de RSSF, importa neste momento apresentar uma vis�o critica do trabalho
relacionado como forma de enquadr�-lo como base te�rica da disserta��o.

Em primeiro lugar pode-se apresentar os ataques que foram estudados e apresent�-los, de forma
estruturada, relacionando-os com as contra-medidas para os mitigar.
%tabela de ataques e contra-medidas
\begin{table}[H]
\advance\leftskip-4cm
\begin{normalsize}
\begin{tabular}[t]{l|l|p{9cm}}
\hline
Modelos & Ataque & Contramedidas\\ \hline\hline
\multirow{1}{*}{Dolev-Yao}
& Ataque ao meio de comunica��o & Criptografia sim�trica, \textit{One Way Hashing} \\\cline{2-3}
\hline 
%
\multirow{4}{*}{Organiza��o e Descoberta da Rede}
 & Falsifica��o de informa��o de Routing & Autentica��o, \textit{One Way Hashing}\\\cline{2-3}
%%
 & Ataques de \textit{Rushing} & Selec��o aleat�ria de RREQ, autentica��o, verifica��o bidirectional
\\ \cline{2-3}
\hline
%
\multirow{4}{*}{Estabelecimento de Rotas}
 & HELLO flooding & Autentica��o com verifica��o bidirectional(\textit{acknowledge})\\\cline{2-3}
%% 
& Ataques \textit{Sinkhole} & Autentica��o, Distribui��o de chaves \textit{pairwise} \\\cline{2-3}
%% 
& Ataques \textit{Wormhole} & \textit{Packet leaches}, MAC\\\cline{2-3}
%% 
& Ataques \textit{Sybil} & Distribui��o de chaves \textit{pairwise}, selec��o aleat�ria de canais
de r�dio \\
 \hline
%
\multirow{1}{*}{Manuten��o de Rotas} & Ataques de \textit{Backhole} & Defini��o de temporizadores e
mecanismos de confirma��o (ACK) autenticados\\
\hline
%
\multirow{2}{*}{Modelo de Intrus�o}
& Intrus�o& Encaminhamento multi-rota; \textit{One Way Hashing} \\\cline{2-3}
%%
& Replica��o& Certifica��o central; Autentica��o; N�s vizinhos como testemunhas\\\cline{2-3}
\hline

\end{tabular} 
\caption{Tabela de Ataques \textit{vs} Contramedidas}\label{tab:tabela_ataques_contramedidas}
\end{normalsize}
\end{table}

No ponto de vista dos protocolos estudados cabe relacionar as capacidades de cada um para fazer
face a ataques definidos no modelo de advers�rio e tipificados nas diferentes fases dos protocolos
em que estes se podem desencadear.
%tabela de ataques e protocolos de encaminhamento
\thispagestyle{empty}

{%
\begin{table}[H]
\advance\leftskip-4cm
\begin{normalsize}
\begin{tabular}{c|c|c|c|c|c|c|c|c|c|}\cline{2-10}
\mc{1}{c}{\textbf{}} & \mc{7}{|c|}{\textbf{Ataques ao Encaminhamento}}&
\mc{1}{|c|}{\textbf{Intrus�o}} & \mc{1}{|c|}{\textbf{Comunica��o}}\\\cline{1-10}
\mc{1}{|c|}{\textbf{Protocolos}} & \mc{1}{|c}{\textbf{Info. Falsa}} &
\mc{1}{|c|}{\textbf{\textit{Rushing}}}& \mc{1}{c|}{\textbf{HELLO flooding}} &
\mc{1}{c|}{\textbf{\textit{Sinkhole}}} & \mc{1}{c|}{\textbf{\textit{Wormhole}}} &
\mc{1}{c|}{\textbf{\textit{Sybil}}}& \mc{1}{c|}{\textbf{\textit{Blackhole}}} &
\mc{1}{c|}{\textbf{\textit{Intrus�o/Replica��o}}} & \mc{1}{c|}{\textbf{\textit{Dolev-Yao}}} \\\hline
%% linhas da tabela
\mc{1}{|c|}{\textbf{SIGF}} & \checkmark & \checkmark &\checkmark & \texttimes &
\mc{1}{c|}{\texttimes} &\checkmark & \mc{1}{c|}{\checkmark} & \texttimes/\texttimes & \checkmark
\\\hline
%
\mc{1}{|c|}{\textbf{INSENS}} & \checkmark & \checkmark & \checkmark & \checkmark  &
\checkmark  & \checkmark & \checkmark  & \checkmark/\texttimes & \checkmark \\\hline
%
\mc{1}{|c|}{\textbf{Clean-Slate}} & \checkmark & \checkmark & \checkmark & \checkmark &
\checkmark & \checkmark& \checkmark & \checkmark/\checkmark & \checkmark\\\hline
\end{tabular}
\caption{Tabela de Protocolos de Encaminhamento \textit{vs} Ataques
}\label{tab:ataques_vs_protocolos}
\end{normalsize}
\end{table}
}%

Por fim e sendo a an�lise dos ambientes de simula��o um dos focos do trabalho relacionado
poder-se-� avaliar de forma comparativa os ambientes seleccionados para estudo comparandos com os
crit�rios pensados como adequados para a avalia��o.
{%
\newcommand{\mcc}{\mc{1}{c|}}
\begin{table}[H]
\centering
\begin{scriptsize}
\begin{tabular}{|c||l||p{1.5cm}|p{1.5cm}|p{1.5cm}|p{1.5cm}|p{2cm}|  }
% use packages: color,colortbl
\cline{3-7}
\mc{1}{c}{}&\mc{1}{c}{}  & \mc{5}{|c|}{\textbf{Ambientes de Simula��o}}\\\hline
\mc{2}{|c|}{\textbf{Crit�rios}} &\mcc{\textbf{Prowler/JProwler}} &\mcc{\textbf{J-Sim}} &
\mcc{\textbf{Freemote}} & \mcc{\textbf{ShoX}} &
\mcc{\textbf{WiSeNet}}\\\hline\hline
\multirow{4}{*}{\begin{footnotesize}\begin{sideways}\textbf{\textit{Software}}\end{sideways}
\end{footnotesize}}
& \textbf{Portabilidade da linguagem} & \mcc{Java}& \mcc{Java} & \mcc{Java} &  \mcc{Java}&
\mcc{Java}\\\cline{2-7}
& {\textbf{C�digo Livre Aberto}} & \mcc{Sim}& \mcc{Sim}& \mcc{Sim}&\mcc{Sim}& \mcc{Sim}\\\cline{2-7}
& {\textbf{Modularidade e extensabilidade}} & \mcc{Sim}& \mcc{Sim (JTcl) }&\mcc{Sim}&
\mcc{Sim}& \mcc{Sim}\\\cline{2-7}
& {\textbf{Documenta��o}} & Apenas coment�rios no c�digo& Papers e On-line &
\mcc{Pouca}&\mcc{Pouca}& \- \\\hline\hline
\multirow{11}{*}{\begin{footnotesize} \begin{sideways} \textbf{Propriedades das
RSSF\   \   \   \   \   \   \   \   \   }\end{sideways}\end{footnotesize}}
& {\textbf{Escalabilidade}} & Aprox. 5000 n�s & Documentado na ordem dos milhares& Documentado na
ordem dos milhares&Foram testados 100 n�s sem sucesso & Na ordem dos milhares \\\cline{2-7}
& {\textbf{Colis�es/Comunica��o}} & Sim, modelo B-MAC & Sim, mas 802.11& Sim, mas muito
simples &Sim, mas 802.11&\mcc{Sim} \\\cline{2-7}
& {\textbf{Gest�o de Energia}} & \mcc{N�o} & \mcc{N�o}& \mcc{N�o}&\mcc{Sim}& \mcc{Sim}\\\cline{2-7}
& {\textbf{Emula��o}} &\mcc{N�o} & \mcc{N�o}&Sim, para plataformas Java
\textit{Based}&\mcc{N�o}&\mcc{N�o}\\\cline{2-7}
& {\textbf{Mobilidade}} &Sim, mas rudimentar & \mcc{Sim}&
\mcc{Sim}&\mcc{Sim}&\mcc{N�o}\\\cline{2-7}
& {\textbf{Visualiza��o}} & Sim, mas s� de visualiza��o da topologia & Existem ferramentas
auxiliares& \mcc{Sim}&Sim, mas n�o em tempo real. Apenas depois de executar a simula��o &\mcc{Sim}
\\\cline{2-7}
& {\textbf{Topologia}} & N�o existe de raiz, pode ser modelado & N�o existe de raiz, pode ser
modelado& N�o existe de raiz, pode ser modelado&Existem alguns de raiz, podendo ser estendido&
Existem de raiz alguns modelos, permitindo a adi��o de mais\\\hline
& {\textbf{Injec��o de ataques/falhas}} & N�o existe & N�o existe & N�o
existe&N�o
existe&Existe\\\hline
& {\textbf{Testes}} & N�o existe & N�o existe & N�o
existe&N�o
existe&Existe\\\hline
& {\textbf{Resultados}} & N�o existe & N�o existe & N�o
existe&N�o
existe&Existe\\\hline
& {\textbf{Configura��es}} & N�o existe & N�o existe & N�o
existe&N�o
existe&Existe\\\hline
\end{tabular}
\caption{Tabela de Crit�rios de Avalia��o \textit{vs} Ambientes de
Simula��o
}\label{tab:Criterios_vs_ambientes}
\end{scriptsize}
\end{table}
}%
